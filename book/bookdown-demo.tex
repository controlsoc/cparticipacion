% Options for packages loaded elsewhere
\PassOptionsToPackage{unicode}{hyperref}
\PassOptionsToPackage{hyphens}{url}
%
\documentclass[
]{book}
\usepackage{lmodern}
\usepackage{amssymb,amsmath}
\usepackage{ifxetex,ifluatex}
\ifnum 0\ifxetex 1\fi\ifluatex 1\fi=0 % if pdftex
  \usepackage[T1]{fontenc}
  \usepackage[utf8]{inputenc}
  \usepackage{textcomp} % provide euro and other symbols
\else % if luatex or xetex
  \usepackage{unicode-math}
  \defaultfontfeatures{Scale=MatchLowercase}
  \defaultfontfeatures[\rmfamily]{Ligatures=TeX,Scale=1}
\fi
% Use upquote if available, for straight quotes in verbatim environments
\IfFileExists{upquote.sty}{\usepackage{upquote}}{}
\IfFileExists{microtype.sty}{% use microtype if available
  \usepackage[]{microtype}
  \UseMicrotypeSet[protrusion]{basicmath} % disable protrusion for tt fonts
}{}
\makeatletter
\@ifundefined{KOMAClassName}{% if non-KOMA class
  \IfFileExists{parskip.sty}{%
    \usepackage{parskip}
  }{% else
    \setlength{\parindent}{0pt}
    \setlength{\parskip}{6pt plus 2pt minus 1pt}}
}{% if KOMA class
  \KOMAoptions{parskip=half}}
\makeatother
\usepackage{xcolor}
\IfFileExists{xurl.sty}{\usepackage{xurl}}{} % add URL line breaks if available
\IfFileExists{bookmark.sty}{\usepackage{bookmark}}{\usepackage{hyperref}}
\hypersetup{
  pdftitle={Participación ciudadana y control social},
  pdfauthor={Nelson Shack},
  hidelinks,
  pdfcreator={LaTeX via pandoc}}
\urlstyle{same} % disable monospaced font for URLs
\usepackage{longtable,booktabs}
% Correct order of tables after \paragraph or \subparagraph
\usepackage{etoolbox}
\makeatletter
\patchcmd\longtable{\par}{\if@noskipsec\mbox{}\fi\par}{}{}
\makeatother
% Allow footnotes in longtable head/foot
\IfFileExists{footnotehyper.sty}{\usepackage{footnotehyper}}{\usepackage{footnote}}
\makesavenoteenv{longtable}
\usepackage{graphicx,grffile}
\makeatletter
\def\maxwidth{\ifdim\Gin@nat@width>\linewidth\linewidth\else\Gin@nat@width\fi}
\def\maxheight{\ifdim\Gin@nat@height>\textheight\textheight\else\Gin@nat@height\fi}
\makeatother
% Scale images if necessary, so that they will not overflow the page
% margins by default, and it is still possible to overwrite the defaults
% using explicit options in \includegraphics[width, height, ...]{}
\setkeys{Gin}{width=\maxwidth,height=\maxheight,keepaspectratio}
% Set default figure placement to htbp
\makeatletter
\def\fps@figure{htbp}
\makeatother
\setlength{\emergencystretch}{3em} % prevent overfull lines
\providecommand{\tightlist}{%
  \setlength{\itemsep}{0pt}\setlength{\parskip}{0pt}}
\setcounter{secnumdepth}{5}
\usepackage{booktabs}
\usepackage{amsthm}
\makeatletter
\def\thm@space@setup{%
  \thm@preskip=8pt plus 2pt minus 4pt
  \thm@postskip=\thm@preskip
}
\makeatother
\usepackage[]{natbib}
\bibliographystyle{apalike}

\title{Participación ciudadana y control social}
\author{Nelson Shack}
\date{2020-07-18}

\begin{document}
\maketitle

{
\setcounter{tocdepth}{1}
\tableofcontents
}
\hypertarget{sobre-el-curso}{%
\chapter*{Sobre el curso}\label{sobre-el-curso}}
\addcontentsline{toc}{chapter}{Sobre el curso}

En el curso ``Participación ciudadana y control social'', se busca que los alumnos conozcan herramientas formales para participar de los asuntos públicos existentes en las normas en la actualidad.

En el primer capítulo, se desarrollarán y explicarán las normas constitucionales y legales que amparan el derecho ciudadana a la participación. Así durante esta primera etapa, nos centraremos en el estudio de instrumentos normativos relevantes para conocer sobre la materia. También se revisarán antecedentes históricos o cómo ha sido abordada la participación ciudadana con anterioridad.

En el segundo capítulo, se desarrollarán los mecanismos de participación ciudadana con los que cuenta la ciudadanía para influir en los asuntos públicos, desde cómo formar parte de la creación de leyes, referendums, mecanismos de acceso a la información pública, entre otros, con la finalidad de que al terminar el capítulo, el alumnado pueda hacer uso de estos mecanismos cuando lo estime conveniente.

En el tercer capítulo, se analizarán dos casos relativos a la participación ciudadana: la revocatoria, y el referendum del FONAVI.

En el cuarto capítulo, se estudiarán sentencias de tribunales nacionales e internacionales con información relevante para el control social.

Finalmente, en el quinto capítulo se brindarán algunas conclusiones y recomendaciones.

Habrán dos anexos, uno con un caso a desarrollar por parte de los y las alumnas; y el otro con las referencias de las fuentes consultadas para la compilación de este material.

\hypertarget{sobre-nelson-shack}{%
\chapter*{Sobre Nelson Shack}\label{sobre-nelson-shack}}
\addcontentsline{toc}{chapter}{Sobre Nelson Shack}

Amplia experiencia directiva en la administración pública. Realizó consultorías internacionales y trabajos para el Banco Interamericano de Desarrollo, Banco Mundial, Delegación de la Comisión Europea y la Comisión Económica para América Latina en más de una docena de países de la Región en diversos temas de Gestión Presupuestaría, Planificación Estratégica, Programación de Inversiones, Auditorias de Desempeño, y Evaluaciones de Gasto Público y Rendición de Cuentas, y Análisis de la calidad del Gasto en áreas de Seguridad y Justicia.

Se ha desempeñado como Coordinador del Proyecto para el Mejoramiento de los servicios de Justicia en el Perú. Ha sido miembro del Consejo Consultivo de la Presidencia del Poder Judicial, Director General de Presupuesto Público, y Director General de Asuntos Económicos y Sociales del Ministerio de Economía y Finanzas, así como Director del Banco de la Nación. Es autor de diversas publicaciones e investigaciones especializadas, y es expositor y docente de Postgrado de la Universidad del Pacífico, Universidad Continental, Universidad César Vallejo, Universidad Esan y de la Pontificia Universidad Católica del Perú.

\hypertarget{antecedentes-normas-y-definiciones-relevantes}{%
\chapter{Antecedentes, normas y definiciones relevantes}\label{antecedentes-normas-y-definiciones-relevantes}}

El máximo órgano constitucional en nuestro país, ha hecho referencia al principio democrático recogido en la Constitución y ha desarrollado también sus presupuestos mínimos, teniendo como base fundamental que la ``aceptación de que la persona humana y su dignidad son el inicio y el fin del Estado'' y que la participación de ésta en la formación de la voluntad político-estatal, es presupuesto indispensable para garantizar el máximo respeto a la totalidad de sus derechos constitucionales. (Expediente 0677-2004-AA/TC. Fundamento 12)

Este desarrollo del principio democrático que como se aprecia, necesita de la participación del ciudadano en la vida de la nación, pero también, para hacer referencia a nuestro sistema democrático se necesita contar con una serie de elementos tales como:

``el reconocimiento de un gobierno representativo y del principio de separación de poderes (artículo 43º de laConstitución), de mecanismos de democracia directa (artículo 31 º de la Constitución), de instituciones políticas (artículo 35° de la Constitución), del principio de alternancia en el poder y de tolerancia; así como de una serie de derechos fundamentales cuya
vinculación directa con la consolidación y estabilidad de una sociedad democrática,
hace de ellos, a su vez, garantías institucionales de ésta''

Todo esto de manera conjunta, representa para el Tribunal Constitucional peruano, una garantía del sistema en el que vivimos en la actualidad.

\hypertarget{introducciuxf3n-a-participaciuxf3n-ciudadana}{%
\section{Introducción a participación ciudadana}\label{introducciuxf3n-a-participaciuxf3n-ciudadana}}

¿A qué se hace referencia cuando se menciona a la participación ciudadana?

En el Perú la participación ciudadana no solo es espontánea, sino que está reconocida en una serie de mecanismos con rango constitucional o legal; pero podríamos separar a estos en cuatro grupos para una mejor comprensión.

La participación ciudadana como medio de la democracia representativa; tanto así que en tratados de derechos humanos, la referencia a la participación estaba también íntimamente ligada al derecho de todo ciudadano de elegir y ser elegido representante.El acto de participar en las elecciones podría ser considerada la definición más clásica de participación.

Por otro lado, la participación ciudadana también comprende mecanismos de participación directa, tales como el referéndum, la revocatoria y la remoción de autoridades.

También, la participación ciudadana, comprende mecanismos de participación en la gestión pública, como el involucramiento en los planes de desarrollo concertado, consejos de coordinación regional y local, y presupuesto participativo.

Además, también como mecanismos de control y proposición, mediante el derecho al acceso a la
información pública,la demanda de rendición de cuentas y las iniciativas legislativas y de reforma constitucional.

Los anteriores, han hecho referencia a mecanismos que tienen un fundamento constitucional de la CPP de 1993; sin embargo, existen otros que encuentran su base en normas de rango legal o sentencias del Tribunal constitucional o mecanismos del derecho internacional, como los datos abiertos, el derecho a la protesta, las audiencias públicas y participación en organismos internacionales.

\hypertarget{antecedentes-y-normas}{%
\section{Antecedentes y normas}\label{antecedentes-y-normas}}

En el derecho internacional de los derechos humanos, la participación ciudadana ha sido entendida, al menos desde el derecho de los tratados, en su versión más clásica ligado a la democracia representativa. No obstante, durante el desarrollo del derecho, la participación de la sociedad civil ha sido fundamental para el avance de los derechos y la participación ciudadana es un pilar de los sistemas de naciones unidas y de la organización de estados americanos.

\begin{quote}
Artículo 21.
1. Toda persona tiene derecho a participar en el gobierno de su país, directamente o por medio de representantes libremente escogidos.2. Toda persona tiene el derecho de acceso, en condiciones deigualdad, a las funciones públicas de su país.3. La voluntad del pueblo es la base de la autoridad del poder público; esta voluntad se expresará mediante elecciones auténticas que habrán de celebrarse periódicamente, por sufragio universal e igual y por voto secreto u otro procedimiento equivalente que garantice la libertad del voto.
\end{quote}

\begin{quote}
-- Declaración Universal de Derechos Humanos (1948)
\end{quote}

\begin{quote}
Artículo 25
Todos los ciudadanos gozarán, sin ninguna de la distinciones mencionadas en el artículo 2, y sin restricciones indebidas, de los siguientes derechos y oportunidades:
Participar en la dirección de los asuntos públicos, directamente o por medio de representantes libremente elegidos;
b) Votar y ser elegidos en elecciones periódicas, auténticas,realizadas por sufragio universal e igual y por voto secreto que garantice la libre expresión de la voluntad de los electores;
c) Tener acceso, en condiciones generales de igualdad, a las funciones públicas de su país.
.
\end{quote}

\begin{quote}
-- Pacto de Derechos Civiles y Políticos (1966)
\end{quote}

\begin{quote}
Artículo 23. Derechos Políticos
1. Todos los ciudadanos deben gozar de los siguientes derechos y oportunidades:
a) de participar en la dirección de los asuntos públicos, directamente o por medio de
representantes libremente elegidos;
b) de votar y ser elegidos en elecciones periódicas auténticas, realizadas por sufragio universal e igual y por voto secreto que garantice la libre expresión de la voluntad de los electores, y
c) de tener acceso, en condiciones generales de igualdad, a las funciones públicas de su país.
\end{quote}

\begin{quote}
-- Convención Americana de Derechos Humanos (1969)
\end{quote}

Si hacemos una revisión histórica de las tres últimas Constituciones Políticas del Perú, encontraremos que:

\begin{quote}
Artículo 84.- Son ciudadanos los peruanos varones mayores de edad, los casados mayores de 18 años y los emancipados.
Artículo 85.- El ejercicio de la ciudadanía se suspende:
1o. Por incapacidad física o mental; 2o. Por profesión religiosa; y 3o. Por ejecución de sentencia que imponga pena privativa de la libertad.
Artículo 86.- Gozan del derecho de sufragio los ciudadanos que sepan leer y escribir; y, en elecciones municipales, las mujeres peruanas mayores de edad, las casadas o que lo hayan llegado a su mayoría.
\end{quote}

\begin{quote}
-- Constitución Política del Perú (1933)
\end{quote}

Caracterizada por : Participación a través del voto solo para algunos.

\begin{quote}
Toda persona tiene derecho a:
16. A participar, en forma individual o asociada, en la vida política, económica, social y cultural de la nación.
64.Los ciudadanos tienen el derecho de participar en los asuntos públicos, directamente o por medio de representantes libremente elegidos en comicios periódicos y de acuerdo con las condiciones determinadas por ley. Es nulo y punible todo acto por el cual se prohíbe o limita al ciudadano o partido intervenir en la vida política de la Nación.
65.­Son ciudadanos los peruanos mayores de dieciocho años. Para el ejercicio de la ciudadanía se requiere estar inscrito en el Registro Electoral. Tienen derecho a votar todos los ciudadanos que están en el goce de su capacidad civil. El voto es personal, igual, libre, secreto y obligatorio hasta los setenta años. Es facultativo después de esta edad. En las elecciones pluripersonales, hay representación proporcional, conforme al sistema que establece la ley.
\end{quote}

\begin{quote}
-- Constitución Política del Perú (1979)
\end{quote}

Si bien la única institución establecida es la del derecho de sufragio, hay un cambio importante en relación a la CPP de 1933, dado que la participación forma parte de una serie de disposiciones constitucionales.

\hypertarget{constituciuxf3n-poluxedtica-del-peruxfa-de-1993}{%
\section{Constitución Política del Perú de 1993}\label{constituciuxf3n-poluxedtica-del-peruxfa-de-1993}}

¿Qué artículos de la Constitución Política hacen referencia a la participación ciudadana?

Sobre participación y control.

\begin{quote}
Artículo 2°.-
"Toda persona tiene derecho :
\end{quote}

\begin{quote}
A solicitar sin expresión de causa la información que requiera y a recibirla
de cualquier entidad pública, en el plazo legal, con el costo que suponga el
pedido. Se exceptúan las informaciones que afectan la intimidad personal
y las que expresamente se excluyan por ley o por razones de seguridad
nacional.
\end{quote}

\begin{quote}
(\ldots) Inciso 17. A participar, en forma individual o asociada, en la vida política, económica,
social y cultural de la Nación. Los ciudadanos tienen, conforme a ley, los
derechos de elección, de remoción o revocación de autoridades, de
iniciativa legislativa y de referéndum.
\end{quote}

\begin{quote}
--- Constitución Política del Perú
\end{quote}

En materia educativa

\begin{quote}
(\ldots) Artículo 13°.- La educación tiene como finalidad el desarrollo integral de la
persona humana. El Estado reconoce y garantiza la libertad de enseñanza. Los
padres de familia tienen el deber de educar a sus hijos y el derecho de escoger
los centros de educación y de participar en el proceso educativo.
\end{quote}

\begin{quote}
--- Constitución Política del Perú
\end{quote}

Participación, control y democracia representativa

\begin{quote}
Artículo 31°. - Los ciudadanos tienen derecho a participar en los asuntos públicos
mediante referéndum; iniciativa legislativa; remoción o revocación de autoridades
y demanda de rendición de cuentas. Tienen también el derecho de ser elegidos y
de elegir libremente a sus representantes, de acuerdo con las condiciones y
procedimientos determinados por ley orgánica.
\end{quote}

\begin{quote}
Es derecho y deber de los vecinos participar en el gobierno municipal de su
jurisdicción. La ley norma y promueve los mecanismos directos e indirectos de su
participación.
\end{quote}

\begin{quote}
Tienen derecho al voto los ciudadanos en goce de su capacidad civil. Para el
ejercicio de este derecho se requiere estar inscrito en el registro correspondiente.
El voto es personal, igual, libre, secreto y obligatorio hasta los setenta años. Es
facultativo después de esa edad.
\end{quote}

\begin{quote}
La ley establece los mecanismos para garantizar la neutralidad estatal durante los
procesos electorales y de participación ciudadana.
\end{quote}

\begin{quote}
Es nulo y punible todo acto que prohíba o limite al ciudadano el ejercicio de sus
derechos.
\end{quote}

\begin{quote}
--- Constitución Política del Perú
\end{quote}

\begin{quote}
Artículo 32°. - Pueden ser sometidas a referéndum:
1. La reforma total o parcial de la Constitución;
2. La aprobación de normas con rango de ley;
3. Las ordenanzas municipales; y
4. Las materias relativas al proceso de descentralización.
\end{quote}

\begin{quote}
No pueden someterse a referéndum la supresión o la disminución de los derechos
fundamentales de la persona, ni las normas de carácter tributario y presupuestal,
ni los tratados internacionales en vigor.
\end{quote}

\begin{quote}
--- Constitución Política del Perú
\end{quote}

Estado Social de derecho

\begin{quote}
Artículo 44°.- Son deberes primordiales del Estado: defender la soberanía
nacional; garantizar la plena vigencia de los derechos humanos; proteger a la
población de las amenazas contra su seguridad; y promover el bienestar general
que se fundamenta en la justicia y en el desarrollo integral y equilibrado de la
Nación.
\end{quote}

\begin{quote}
Asimismo, es deber del Estado establecer y ejecutar la política de fronteras y
promover la integración, particularmente latinoamericana, así como el desarrollo y
la cohesión de las zonas fronterizas, en concordancia con la política exterior.
\#\# Leyes emitidas por el Congreso de la República.
\end{quote}

\begin{quote}
--- Constitución Política del Perú
\end{quote}

Sobre Gobiernos Regionales y Municipales

\begin{quote}
Artículo 191°.- Los gobiernos regionales tienen autonomía política, económica y
administrativa en los asuntos de su competencia. Coordinan con las
municipalidades sin interferir sus funciones y atribuciones.
La estructura orgánica básica de estos gobiernos la conforman el Consejo
Regional como órgano normativo y fiscalizador, el Presidente como órgano
ejecutivo, y el Consejo de Coordinación Regional integrado por los alcaldes
provinciales y por representantes de la sociedad civil, como órgano consultivo y
de coordinación con las municipalidades, con las funciones y atribuciones que les
señala la ley.
\end{quote}

\begin{quote}
Artículo 197°.- Las municipalidades promueven, apoyan y reglamentan la
participación vecinal en el desarrollo local. Asimismo brindan servicios de
seguridad ciudadana, con la cooperación de la Policía Nacional del Perú,
conforme a ley.
\end{quote}

\begin{quote}
Artículo 199°.- Los gobiernos regionales y locales son fiscalizados por sus
propios órganos de fiscalización y por los organismos que tengan tal atribución
por mandato constitucional o legal, y están sujetos al control y supervisión de la
Contraloría General de la República, la que organiza un sistema de control
descentralizado y permanente. Los mencionados gobiernos formulan sus
presupuestos con la participación de la población y rinden cuenta de su ejecución,
anualmente, bajo responsabilidad, conforme a ley.
\end{quote}

\begin{quote}
--- Constitución Política del Perú
\end{quote}

\hypertarget{leyes-sobre-participaciuxf3n-ciudadana}{%
\section{Leyes sobre participación ciudadana}\label{leyes-sobre-participaciuxf3n-ciudadana}}

\begin{quote}
Artículo 2.- Derechos de participación ciudadana
Son derechos de participación de los ciudadanos los siguientes:
a) Iniciativa de Reforma Constitucional;
b) Iniciativa en la formación de las leyes;
c) referéndum;
d) iniciativa en la formación de ordenanzas regionales y ordenanzas municipales; y,
e) otros mecanismos de participación establecidos en la legislación vigente.
\end{quote}

\begin{quote}
Artículo 3.- Derechos de control ciudadano
Son derechos de control de los ciudadanos los siguientes:
a) Revocatoria de Autoridades,
b) Remoción de Autoridades;
c) Demanda de Rendición de Cuentas; y,
d) Otros mecanismos de control establecidos por la presente ley para el ámbito de los gobiernos municipales y regionales.
\end{quote}

\begin{quote}
--- Ley de los derechos de participación y control ciudadanos, Ley N° 26300
\end{quote}

\begin{quote}
Artículo 32.- La gestión de Gobierno Regional se rige por el Plan de Desarrollo Regional Concertado de mediano y largo plazo, así
como el Plan Anual y el Presupuesto Participativo Regional, aprobados de conformidad con políticas nacionales y en cumplimiento
del ordenamiento jurídico vigente.
\end{quote}

\begin{quote}
--- Ley Orgánica de Gobiernos Regionales
\end{quote}

\begin{quote}
Artículo 112.- PARTICIPACIÓN VECINAL
Los gobiernos locales promueven la participación vecinal en la formulación, debate y concertación de sus planes de desarrollo,
presupuesto y gestión. Para tal fin deberá garantizarse el acceso de todos los vecinos a la información.
\end{quote}

\begin{quote}
--- Ley Orgánica de Municipalidades
\end{quote}

\hypertarget{definiciones-relevantes}{%
\section{Definiciones relevantes}\label{definiciones-relevantes}}

\hypertarget{mecanismos-de-participaciuxf3n-directa}{%
\subsection{Mecanismos de participación directa}\label{mecanismos-de-participaciuxf3n-directa}}

\begin{enumerate}
\def\labelenumi{\arabic{enumi}.}
\tightlist
\item
  Referéndum:
\end{enumerate}

Es un mecanismo de participación ciudadana por el que se le consulta a la población su opinión
acerca de temas considerados de especial importancia para el desarrollo y progreso de un país.
Pueden ser sometidas a referéndum:

\begin{quote}
La reforma total o parcial de la Constitución.
La aprobación de normas con rango de ley.
Las ordenanzas municipales.
Las materias relativas al proceso de descentralización.
No pueden someterse a referéndum:
La supresión o la disminución de los derechos fundamentales de la persona.
Las normas de carácter tributario y presupuestal.
Los tratados internacionales en vigor.
\end{quote}

\begin{enumerate}
\def\labelenumi{\arabic{enumi}.}
\setcounter{enumi}{1}
\tightlist
\item
  Revocatoria de autoridades
\end{enumerate}

Consiste en un proceso de elecciones en el que el ciudadano participa directamente, con su voto, para separar de sus cargos a las autoridades regionales, municipales provinciales o distritales que eligió.
Pueden ser objeto de revocación las siguientes autoridades:

\begin{quote}
Alcaldes y regidores provinciales.
Alcaldes y regidores distritales.
Presidentes, vicepresidentes y consejeros regionales.
\end{quote}

\begin{enumerate}
\def\labelenumi{\arabic{enumi}.}
\setcounter{enumi}{2}
\tightlist
\item
  Remoción de auoridades
\end{enumerate}

Proceso para destituir autoridades designadas por el gobierno central o regional.

\hypertarget{mecanismos-de-participaciuxf3n-en-la-gestion-puxfablica}{%
\subsection{Mecanismos de Participación en la gestion pública}\label{mecanismos-de-participaciuxf3n-en-la-gestion-puxfablica}}

\begin{enumerate}
\def\labelenumi{\arabic{enumi}.}
\tightlist
\item
  Consejos de coordinación regional y local:
\end{enumerate}

Órganos de coordinación y concertación de los gobiernos regionales y locales necesarios para la elaboración del Plan de Desarrollo y Presupuesto Participativo.

\hypertarget{mecanismos-de-participaciuxf3n-de-control-y-proposiciuxf3n}{%
\subsection{Mecanismos de Participación de control y proposición}\label{mecanismos-de-participaciuxf3n-de-control-y-proposiciuxf3n}}

\begin{enumerate}
\def\labelenumi{\arabic{enumi}.}
\tightlist
\item
  Derecho al acceso a la información pública
\end{enumerate}

Toda persona tiene derecho a solicitar información sin expresión de causa; y las entidades de la Administración pública tienen la obligación de entregar la información requerida que se encuentre bajo su control.

\begin{enumerate}
\def\labelenumi{\arabic{enumi}.}
\setcounter{enumi}{1}
\tightlist
\item
  Rendición de cuentas
\end{enumerate}

Derecho a interpelar a las autoridades acerca de la ejecución presupuestal.

\begin{enumerate}
\def\labelenumi{\arabic{enumi}.}
\setcounter{enumi}{2}
\tightlist
\item
  Iniciativa legislativa
\end{enumerate}

Se pueden presentar más de un proyecto de ley. No procede en temas tributarios o presupuestarios.

\begin{enumerate}
\def\labelenumi{\arabic{enumi}.}
\setcounter{enumi}{3}
\tightlist
\item
  Iniciativa en la formación de ordenanzas municipales
\end{enumerate}

Derecho mediante el cual, los vecinos plantean al gobierno local una ordenanza municipal

\begin{enumerate}
\def\labelenumi{\arabic{enumi}.}
\setcounter{enumi}{4}
\tightlist
\item
  Reforma constitucional
\end{enumerate}

Plantear una reforma total o parcial a la Constitución. Improcedente para recorte de derechos fundamentales

\hypertarget{mecanismos-de-participaciuxf3n-ciudadana-y-control-social}{%
\chapter{Mecanismos de Participación Ciudadana y Control Social}\label{mecanismos-de-participaciuxf3n-ciudadana-y-control-social}}

\hypertarget{iniciativa-de-reforma-constitucional}{%
\section{Iniciativa de Reforma Constitucional}\label{iniciativa-de-reforma-constitucional}}

\hypertarget{iniciativa-de-formaciuxf3n-de-leyes}{%
\section{Iniciativa de formación de leyes}\label{iniciativa-de-formaciuxf3n-de-leyes}}

\hypertarget{referendum}{%
\section{Referendum}\label{referendum}}

\hypertarget{cuxf3mo-participar-en-el-proceso-de-creaciuxf3n-de-ordenanzas-municipalesregionales}{%
\section{¿Cómo participar en el proceso de creación de ordenanzas municipales/regionales?}\label{cuxf3mo-participar-en-el-proceso-de-creaciuxf3n-de-ordenanzas-municipalesregionales}}

\hypertarget{revocatoria-de-autoridades}{%
\section{Revocatoria de Autoridades}\label{revocatoria-de-autoridades}}

Consiste en el mecanismo que permite destituir de sus cargos a autoridades elegidas como:
\textgreater Presidente, Vicepresidente y Consejeros Regionales
\textgreater Alcaldes y Regidores (Provinciales o Distritales)
\textgreater Jueces de paz que provengan de elección popular

\hypertarget{remociuxf3n-de-autoridades}{%
\section{Remoción de autoridades}\label{remociuxf3n-de-autoridades}}

Consiste en el mecanismo para destituir a las autoridades designadas por el Gobiernos Central:
\textgreater Gobernador.
\textgreater Teniente gobernador.
\#\# Demanda de Rendición de Cuentas

\hypertarget{acceder-a-informaciuxf3n-puxfablica-y-portales-de-transparencia}{%
\section{Acceder a información pública y portales de transparencia}\label{acceder-a-informaciuxf3n-puxfablica-y-portales-de-transparencia}}

El artículo 2, inciso 5 de la Constitución Política del Perú establece que toda persona tiene derecho a solicitar información sin expresión de causa; así también,
el artículo 10 del TUO de la Ley de Transparencia y Acceso a la Información Pública establece la obligación de las entidades de la Administración pública de entregar información requerida que se encuentre bajo su control.

Entonces, ¿cómo efectivizamos este derecho?

El primer paso, es presentar tu solicitud de acceso a la información pública. Esta puede ser presentada en vía virtual o física.

El segundo paso, es esperar que la entidad conteste este pedido. Las entidades tienen 10 días hábiles para atender las solicitudes. En algunos casos, de acuerdo a la norma en casos complejos, la entidad hasta el día 2 de recibida la solicitud, podrá solicitar una prórroga - no existiendo un plazo para contestar por parte de la entidad.

En el tercer paso se pueden dar hasta tres situaciones: a) recibir la información de la entidad, b)que la entidad deniegue o no conteste la solicitud del ciudadano y c) que la entidad entregue información incompleta o distinta a la solicitada.

Si estamos frente al supuesto a) no hay ningún problema.

Si la entidad, no contestara la solicitud, supuesto b), el ciudadano puede apelar esta no respuesta o estime conveniente o supuesto c) cuando la entidad entregue información incompleta o distinta a la solicitada, el ciudadano tiene hasta 15 días para apelar esa decisión de la administración ante el Tribunal de Tranparencia o ante la entidad; o si decidiera acudir al poder judicial a través de un habeas data hasta 60 días después de notificada la decisión de la entidad.

Si nos centramos en el procedimiento administrativo ante el Tribunal de Transparencia, este tiene 10 días después de admitida la apelación por parte del Tribunal de Transparencia, este tendrá diez días para evaluar el caso, pedir a la entidad la información correspondiente y emitir una resolución para que la entidad entregue la información.

Sin embargo, si la entidad no entrega la información, el Tribunal lo único que hace es remitir un reiterativo; dado que no tiene capacidad sancionatoria y se agota la vía administrativa, por lo que se puede acudir a la vía judicial o al OCI de las entidades para solicitar la sanción de los funcionarios responsables.

Si se decide interponer un recurso de habeas data, \ldots{}

\hypertarget{datos-abiertos-de-las-entidades}{%
\section{Datos abiertos de las entidades}\label{datos-abiertos-de-las-entidades}}

\hypertarget{sentencias}{%
\chapter{Sentencias}\label{sentencias}}

\hypertarget{tribunales-nacionales}{%
\section{Tribunales Nacionales}\label{tribunales-nacionales}}

\hypertarget{tribunales-internacionales}{%
\section{Tribunales Internacionales}\label{tribunales-internacionales}}

\hypertarget{conclusiones-y-recomendaciones}{%
\chapter{Conclusiones y recomendaciones}\label{conclusiones-y-recomendaciones}}

\hypertarget{caso-de-estudio}{%
\chapter*{Caso de estudio}\label{caso-de-estudio}}
\addcontentsline{toc}{chapter}{Caso de estudio}

  \bibliography{book.bib,packages.bib}

\end{document}
